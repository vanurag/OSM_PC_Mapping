% ETH Zurich  - 3D Photography 2015
% http://www.cvg.ethz.ch/teaching/3dphoto/
% Template for project proposals

\documentclass[11pt,a4paper,oneside,onecolumn]{IEEEtran}
\usepackage{graphicx}
\usepackage{hyperref}
\usepackage[x11names]{xcolor}
\usepackage{colortbl}
\usepackage{caption}
% \usepackage{fourier, heuristica}
\usepackage{array, booktabs}
\DeclareCaptionFont{blue}{\color{LightSteelBlue3}}
\newcommand{\foo}{\color{LightSteelBlue3}\makebox[0pt]{\textbullet}\hskip-0.5pt\vrule width 1pt\hspace{\labelsep}}
% Enter the project title and your project supervisor here
\newcommand{\ProjectTitle}{Mapping Pointclouds to OSM Building Outlines}
\newcommand{\ProjectSupervisor}{Torsten Sattler}
\newcommand{\DateOfReport}{March 6, 2015}

% Enter the team members' names and path to their photos. Comment / uncomment related definitions if the number of members are different than 2.
% Including photographs are optional. Photos are there to help us to evaluate your group more effectively. If you wish not to include your photos, please comment the following line.
\newcommand{\PutPhotos}{}
% Please include a clear photo of each member. (use pdf or png files for Latex to embed them in the document well)
\newcommand{\memberone}{Anurag Sai Vempati}
\newcommand{\memberonepicture}{pic1.png}
\newcommand{\membertwo}{Wolf Vollprecht}
\newcommand{\membertwopicture}{images/wolf.jpg}
%\newcommand{\memberthree}{Member Name}
%\newcommand{\memberthreepicture}{pic3.png}


%%%% DO NOT EDIT THE PART BELOW %%%%
\title{\ProjectTitle}
\author{\memberone, \membertwo\\Supervised by: \ProjectSupervisor\\ \DateOfReport}
\begin{document}
\maketitle
\vspace{-1.5cm}\section*{Group Members}\vspace{0.3cm}
\begin{center}\begin{minipage}{\linewidth}\begin{center}
\begin{minipage}{3 cm}\begin{center}\memberone\ifdefined\PutPhotos\\\vspace{0.2cm}\includegraphics[height=3cm]{\memberonepicture}\fi\end{center}\end{minipage}
\ifdefined\membertwo\begin{minipage}{3 cm}\begin{center}\membertwo\ifdefined\PutPhotos\\\vspace{0.2cm}\includegraphics[height=3cm]{\membertwopicture}\fi\end{center}\end{minipage}\fi
\ifdefined\memberthree\begin{minipage}{3 cm}\begin{center}\memberthree\ifdefined\PutPhotos\\\vspace{0.2cm}\includegraphics[height=3cm]{\memberthreepicture}\fi\end{center}\end{minipage}\fi
\end{center}\end{minipage}\end{center}\vspace{0.3cm}
%%%% END OF PROTECTED LINES %%%%


%%%% BEGIN WRITING THE DOCUMENT HERE %%%%

\section{Description of the project}

The project aims to map pointcloud data of outdoor city environments to \textit{OpenStreetMap}\footnote{\url{osm.org}} (OSM) building outlines.
The pointcloud data is generated by using the Structure-from-Motion technique to extract 3D data from multiple photographs taken from different viewpoints. The photographs are taken by consumer grade cameras and processed by software like VisualSFM.

To enrich the pointcloud data we want to map the facade outlines of the pointcloud to rich data that we get from OpenStreetMaps. This will allow us to identify, for example, shops and to tag different houses which might not work correctly without OSM data.

Furthermore, it is in the scope of the project to automatically extract building heights from the pointcloud data. Once the data is aligned to OSM, it will be possible to get absolute lengths from the pointcloud data. As OSM is a collaborative effort, the software might enable us to contribute back the building height data.


\section{Work packages and timeline}

The idea for matching the outlines is to break it down to a 2D problem and use the Iterative Closest Point technique to find the best match between the building outlines and the pointcloud. For that, the pointcloud will be reduced to a two dimensional pointcloud and the building outlines will be discretized to a number of points (i.e. a pointcloud will be interpolated from the polygons that are existing in OSM).
A state-of-the-art ICP, such as libpointmatcher~\cite{Pomerleau12comp} will then be used to align those two pointclouds.

Investigations will be made on how well the complexity fof OSM data can be reduced i.e., by removing holes and separations between buildings.

The first workpackage will be to use arbitrary GPS coordinates, build a simple interface to input data and automatically download an OSM file.
\begin{table}
\center
\renewcommand\arraystretch{1.4}\arrayrulecolor{LightSteelBlue3}
\captionsetup{singlelinecheck=false, font=blue, labelfont=sc, labelsep=quad}
% \caption{Timeline}\vskip -1.5ex
\begin{tabular}{@{\,}r <{\hskip 2pt} !{\foo} >{\raggedright\arraybackslash}p{5cm}}
% \toprule
\addlinespace[1.5ex]
1947 & AT and T Bell Labs develop the idea of cellular phones\\
1968 & Xerox Palo Alto Research Centre envisage the 'Dynabook\\
1971 & Busicom 'Handy-LE' Calculator\\
1973 & First mobile handset invented by Martin Cooper\\
1978 & Parker Bros. Merlin Computer Toy\\
1981 & Osborne 1 Portable Computer\\
1982 & Grid Compass 1100 Clamshell Laptop\\
1983 & TRS-80 Model 100 Portable PC\\
1984 & Psion Organiser Handheld Computer\\
1991 & Psion Series 3 Minicomputer\\
\end{tabular}
\end{table}

\section{Outcomes and Demonstration}

Give detailed information on the expected outcome of your project and the experiments you plan to test your implementation. If applicable, describe the online or offline demo you plan to present at the end of the semester.




\vspace{1cm}
\textbf{Instructions:}

\begin{itemize}
\item The document should not exceed two pages including the references.
\item Please name the document \textbf{3DPhoto\_Proposal\_Surname1\_Surname2.pdf} and send it to Ya\u{g}{\i}z in an email titled \textbf{[3DPhoto] Project Proposal - Surname1 Surname2}, filling in your surnames.
\end{itemize}

{%\singlespace
{\small
\bibliography{refs}
\bibliographystyle{plain}}}




\end{document}
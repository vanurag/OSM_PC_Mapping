% ETH Zurich  - 3D Photography 2015
% http://www.cvg.ethz.ch/teaching/3dphoto/
% Template for project proposals

\documentclass[11pt,a4paper,oneside,onecolumn]{IEEEtran}
\usepackage{graphicx}
% Enter the project title and your project supervisor here
\newcommand{\ProjectTitle}{Mapping Pointclouds to OSM Building Outlines}
\newcommand{\ProjectSupervisor}{Torsten Sattler}
\newcommand{\DateOfReport}{March 6, 2015}

% Enter the team members' names and path to their photos. Comment / uncomment related definitions if the number of members are different than 2.
% Including photographs are optional. Photos are there to help us to evaluate your group more effectively. If you wish not to include your photos, please comment the following line.
\newcommand{\PutPhotos}{}
% Please include a clear photo of each member. (use pdf or png files for Latex to embed them in the document well)
\newcommand{\memberone}{Anurag Sai Vempati}
\newcommand{\memberonepicture}{anurag.png}
\newcommand{\membertwo}{Member Name}
\newcommand{\membertwopicture}{pic2.png}
%\newcommand{\memberthree}{Member Name}
%\newcommand{\memberthreepicture}{pic3.png}


%%%% DO NOT EDIT THE PART BELOW %%%%
\title{\ProjectTitle}
\author{3D Photography Project Proposal\\Supervised by: \ProjectSupervisor\\ \DateOfReport}
\begin{document}
\maketitle
\vspace{-1.5cm}\section*{Group Members}\vspace{0.3cm}
\begin{center}\begin{minipage}{\linewidth}\begin{center}
\begin{minipage}{3 cm}\begin{center}\memberone\ifdefined\PutPhotos\\\vspace{0.2cm}\includegraphics[height=3cm]{\memberonepicture}\fi\end{center}\end{minipage}
\ifdefined\membertwo\begin{minipage}{3 cm}\begin{center}\membertwo\ifdefined\PutPhotos\\\vspace{0.2cm}\includegraphics[height=3cm]{\membertwopicture}\fi\end{center}\end{minipage}\fi
\ifdefined\memberthree\begin{minipage}{3 cm}\begin{center}\memberthree\ifdefined\PutPhotos\\\vspace{0.2cm}\includegraphics[height=3cm]{\memberthreepicture}\fi\end{center}\end{minipage}\fi
\end{center}\end{minipage}\end{center}\vspace{0.3cm}
%%%% END OF PROTECTED LINES %%%%


%%%% BEGIN WRITING THE DOCUMENT HERE %%%%

\section{Description of the project}

A high level description of the project, mentioning the main goal, the input and planned output data. Typically 4-5 sentences, also citing immediately related literature \cite{paper}.

\section{Work packages and timeline}

The project has been broken down to two basic tasks. The implementation details and the expected outcomes are as follows:

\subsection{Segmenting the Point Cloud:-}
We start with parsing the point cloud data from the bundler files ~\cite{bundler} and use point cloud library ~\cite{pcl} to do the further processing. For getting a 2-D outline of the buildings, we need to be able to segment out the point clouds into parts that belong to a plane which is up-right (and hence shows up as a line in the 2-D map) and the parts that are outliers to these planes. \\

To be able to do this, we first estimate the up-vector which is coplanar with the facades of the building and is normal to the ground. The data we will be working upon, mostly includes pictures taken from ground level. So, the smallest eigenvector of the covariance matrix generated from the camera positions should be vertical to the ground and a reasonable approximation for the up-right vector. Even in the cases where we might not have a fully flat ground, one could break it down into piece-wise flat areas and estimate up-right vectors at each of these regions. \\

Next thing to do would be to find the normal vector corresponding to each 3-D point by taking a patch in the point cloud around each of these points and calculating the smallest eigenvector of the covariance matrix obtained from the points belonging to this patch. The direction of the normal vectors can be used to estimate the probability of each point being normal to the up-right vector. \\

Thresholding upon this probability measure helps in segmenting out the outliers and get the part of the point cloud that can be used to generate a 2-D outline that can be used in the next phase. One could also iteratively refine the point clouds using techniques like RANSAC ~\cite{fischler1981random} or Expectation Maximization ~\cite{dempster1977maximum}.  \\

Anurag will be working on this part and plans to build a catkin package in C++ which can run on 64-bit Linux machines.


\section{Outcomes and Demonstration}

Once fully implemented, our algorithm should be able to efficiently generate the 2-D outline from the dense point cloud and find association between this outline and the one that is obtained from OpenStreetmaps. At the end of the semester, we plan to give a demonstration of the quality of the 2-D outline obtained and the robustness of the matching algorithm to associate the ouline with OSM despite missing information, incomplete maps, noisy measurements and GPS readings.\\

If time permits, we also plan to record our own data from higher altitudes and generate point clouds with much denser building rooftops. Our approach can then be extended to associate this rooftop information to some selected rooftop designs.


\vspace{1cm}
\textbf{Instructions:}

\begin{itemize}
\item The document should not exceed two pages including the references.
\item Please name the document \textbf{3DPhoto\_Proposal\_Surname1\_Surname2.pdf} and send it to Ya\u{g}{\i}z in an email titled \textbf{[3DPhoto] Project Proposal - Surname1 Surname2}, filling in your surnames.
\end{itemize}

{%\singlespace
{\small
\bibliography{refs}
\bibliographystyle{plain}}}




\end{document}
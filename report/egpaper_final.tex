\documentclass[10pt,twocolumn,letterpaper]{article}

\usepackage{cvpr}
\usepackage{times}
\usepackage{epsfig}
\usepackage{graphicx}
\usepackage{amsmath}
\usepackage{amssymb}
\usepackage{hyperref}

% Include other packages here, before hyperref.

% If you comment hyperref and then uncomment it, you should delete
% egpaper.aux before re-running latex.  (Or just hit 'q' on the first latex
% run, let it finish, and you should be clear).
%\usepackage[pagebackref=true,breaklinks=true,letterpaper=true,colorlinks,bookmarks=false]{hyperref}

\cvprfinalcopy % *** Uncomment this line for the final submission

\def\cvprPaperID{****} % *** Enter the 3DV Paper ID here
\def\httilde{\mbox{\tt\raisebox{-.5ex}{\symbol{126}}}}

% Pages are numbered in submission mode, and unnumbered in camera-ready
%\ifcvprfinal\pagestyle{empty}\fi
%\setcounter{page}{4321}
\begin{document}

%%%%%%%%% TITLE
\title{Mapping Building Outlines to 3D Point Clouds}

\author{Anurag Sai Vempati\\
Autonomus Systems Lab\\ ETH Zurich\\
{\tt\small firstauthor@i1.org}
% For a paper whose authors are all at the same institution,
% omit the following lines up until the closing ``}''.
% Additional authors and addresses can be added with ``\and'',
% just like the second author.
% To save space, use either the email address or home page, not both
\and
Wolf Vollprecht\\
\\
\\
{\tt\small wolfv@student.ethz.ch}
}

\maketitle
%\thispagestyle{empty}

%%%%%%%%% ABSTRACT
\begin{abstract}
  While point clouds are easy to obtain through processes such as Structure-From-Motion they only exist in a space for themselves. GPS coordinates, recorded at capture time of the images, can give a sparse and noisy reference to the true position of the point cloud. In this paper we present a method to register a point cloud on reference map data.
\end{abstract}

%%%%%%%%% BODY TEXT
\section{Introduction}

With current state of  the art technology, obtaining point clouds through structure from motion has become an easy task. 
Several attempts deal with a similar problem, such as ... 

\section{Obtaining the ground truth}

\begin{figure}[h]
   \centering
   \includegraphics[width=\linewidth]{images/OSM_Mashup.png}
   \caption{OSM overlaid with editor interface}
   \label{fig:figure1}
\end{figure}


In order to obtain the ground truth building outlines, we used OpenStreetMaps\footnote{\url{http://www.openstreetmap.org/}}, a collaborative mapping effort that takes place globally. Volunteers are mapping their surroundings and upload it to a central database, where all changes to the map are stored. 
Of concern for us is only one mapped entity, the building. All points in OSM are stored as nodes. The entity consists of one or more node (in the case of a building it's multiple nodes). By parsing the XML response and mapping all node values to the corresponding building entity, we can obtain all corner points of the closed polygon that describes a building outline in the real world.

In order to adjust the point cloud to the obtained ground truth, we discretized the polygons to point clouds themselves by creating points along the polygon edges in a certain distance. During the Iterative Closest Point matching these points will be matched against the point cloud and provide an error distance.

The OSM data is in a coordinate format consisting of a latitude/longitude pair. This representation maps coordinates to the Globe, which has a spherical shape. However, we would like to work on a 2D surface. The size of the registration is reasonably small to obtain a 2D representation that falls into the boundaries of acceptable error.

There are a number of methods on how to convert coordinates to a 2D representation, which is a topic that has been explored for a long time since large-scale mapping has taken place globally. The methods can be divided into two different categories:

\begin{itemize}
   \item \textbf{Conformal} This category of mappings preserves angles as observed in the real world (ie. the local angles are preserved). 
   \item \textbf{Equiareal} Retains equal area
   \item \textbf{Equidistant} Preserving distance
   \item \textbf{Azimuthal} Preserving Direction
   \item \textbf{Shortest Route} Shortest route is observable 
\end{itemize}

Since the sphere cannot be flattened out without distortion (ie. the sphere is a non-developable surface) not all of the above properties can hold for a projection.

For our application of registering a 2D point cloud on the map, we chose to use the Universal Transverse Mercator projection, which preserves the angles and shapes, a property we deemed useful for our application.

\section{Registering the Point Cloud}

After obtaining the ground truth and the 2D projection of the 3D cloud we are now able to register the point cloud on the OSM reference.

Therefore we use ``Iterative Closest Point'' (ICP) matching algorithm. As the name suggest, ICP is an iterative algorithm to minimize the distance between two (or more) point cloud measurements. ICP is a widely used algorithm for example in robotics (Simultaneous Localization and Mapping [SLAM]). 

The input to the ICP algorithm in general is two pointclouds, one called the reference and the other source. The source should be aligned to the reference cloud. The output of the algorithm is a transformation matrix in 2 or 3 dimensions. Furthermore, the algorithm needs to be supplied with a stop condition, for example a maximum number of iterations or a minimal size of change between iterations, which indicates that the algorithm has converged to a minima.

Several important drawbacks of ICP are that it usually only provides rigid transformations (i.e. scale and shear factors are not affected). Allowing infinite scale the ICP solution would scale down to only one point with a distance of zero, as all points would be incident with that one. Another problem is that ICP, without good initialisation, tends to convertge to a local minima. Therefore it is paramount to have an initial rotation and translation that matches the reference somewhat. % We are currently able to do this based on the GPS positions in our source data which indicate a 


Some important limitation are that it can only perform rigid transformations.



\end{document}
